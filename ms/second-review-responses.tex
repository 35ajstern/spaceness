%%%%%%
%%
%%  Don't reorder the reviewer points; that'll mess up the automatic referencing!
%%
%%%%%

\begin{minipage}[b]{2.5in}
  Resubmission Cover Letter \\
    {\it Genetics}
\end{minipage}
\hfill
\begin{minipage}[b]{2.5in}
    C.~J.~Battey, \\
    Peter Ralph, \\
    \emph{and} Andrew Kern \\
  \today
\end{minipage}
 
\vskip 2em
 
\noindent
{\bf To the Editor(s) -- }
 
\vskip 1em

We are writing to submit another revised version of our manuscript, 
``Space is the Place: Effects of Continuous Spatial Structure on Analysis of Population Genetic Data''.


\vspace{5em}

\noindent \hspace{4em}
\begin{minipage}{3in}
\noindent
{\bf Sincerely,}

\vskip 2em

{\bf 
C.~J.~Battey, Peter Ralph, and Andrew Kern
}\\
\end{minipage}

\vskip 4em

\pagebreak
\setcounter{page}{1}

%%%%%%%%%%%%%%
\reviewersection{Decision Letter}

\begin{quote}
    It is most important that you address the following in a revised manuscript:
    \begin{enumerate}
        \item in response to reviewer 3, add text to help clarify how the reproduction scheme / dispersal model differs from models that do not induce the bimodality in the distance distribution the reviewer points out (see my comments below as well); 
        \item Respond to the concern about the definitions of sigma and neighborhood size; 
        \item Reviewer 2 has several minor comments that would be straightforward and helpful to address/add depth into your discussion (e.g., comments 2, 4, 5). 
        \item The other minor comments from each reviewer. 
    \end{enumerate}
\end{quote}

Thanks very much for the useful input.
The inaccuracy in describing dispersal under the model was our mistake,
and we have added more careful discussion of this point (see below).
We have modified the manuscript to incorporate all remaining feedback,
as enumerated below,
and think this has improved the manuscript.

\reviewersection{AE Comments}

\begin{quote}
The point of R3 on the bimodality in the realized dispersal distance in your model is interesting. You can make this more explicit in the paper and describe how there are realistic biological scenarios that motivate such a model (many/most plants) and demarcate how well you think you can generalize to others (the new appendix already helps in that regard). It strikes me your model has relevance for patrilocal/matrilocal systems where the paternal/maternal parent-offspring distances differ - though your model is hermaphroditic so it's an imperfect analogy. Perhaps, to explore more typical animal mating patterns, you can quickly check the impact for one or two outcomes of changing the simulation such that offspring are dispersed around the midpoint of the parental origins, as this would remove the bimodality. That said, please be sure we respect the work done is extensive and don't mean to invite/require a complete re-working. The core tension is between ensuring your results are generalizable if that is what you wish to claim; versus specific to a particular model that is biologically motivated. It's particularly key for the material on GWAS, where you are explicitly trying to address human GWAS results: those sections will be read and taken at face value by a large community even if this simulation scheme introduces some discrepancies from what might more realistically emerge. So at a minimum, some caveats directly in the GWAS discussion section would help.
\end{quote}


Thank you for your comments. We have responded to each reviewer below. In response to reviewer 2 we have added several caveats to the GWAS section noting other sources of error such as allele frequency and LD variation among populations. Reviewer 3 raised two good points on our definition of neighborhood size and the interpetation of $\sigma$ -- first, that $\sigma$ refers to one- rather than two-dimensional distances between parent and offspring (this is consistent with previous theoretical work, and we have corrected our definition in the text), and second, that the use of separate draws for mate selection and dispersal means the total distance traveled by a locus along a pedigree link is greater for ``paternal'' than ``maternal'' material (taking ``maternal'' to mean the parent from which dispersal is calculated). Because our simulated genome is autosomal and individuals hermaphroditic and non-selfing, a locus will experience a mix of ``maternal'' and ``paternal'' transmission events along its genealogy. So we see this as a scaling issue rather than one of a ``bimodal'' dispersal kernal per se -- the expected axial displacement of a locus along a pedigree link is $\sqrt{3/2} \sigma$, rather than $\sigma$.
(Note that although the parent-offspring displacement is a mixture of mean-zero Gaussians
with respect to a bimodal distribution of standard deviations,
the norm of any such mixtures is never itself bimodal.)
% However we note that in terms of summary statistics the effects of scaling $\sigma$ by $\sqrt{3/2}$ are weaker than those generated by factors like changing landscape resolution in lattice vs. continuous models (e.g. Appendix 1), so we don't think this significantly changes our interpretation of model outputs.
We have added a paragraph discussing these points (while trying to keep out of the weeds),
to \llname{ll:sigma_effective}.



%%%%%%%%%%%%%%
\reviewersection{1}


\begin{quote}
Only minor changes were needed, and they were made. This study neatly summarises many biases arising from unmodelled continuous space (pertinent for real studies), and I would not extend it by further analyses. When polishing the final version, the authors can make some of their messages stronger. For example, Figure 7 and the accompanying text may give the impression that we are doing pretty well with smc++. It is only when panel B is seen one can appreciate the degree of uncertainty. So, in a way, Figure S7 is more telling as to how much we can trust our results in real analysis. After all, in reality, as authors mention, we only have one sample set, not replicates.
\end{quote}

Thank you for your comments. While we agree that our analysis found problems with demographic inference under isolation by distance, since this manifests as higher variance in predictions rather than a straightforward bias as we expected (i.e. a reduction in recent inferred Ne), we have chosen to keep the somewhat more nuanced presentation of results in that section used in a previous version,
although we added an additional note about the between-replicate variance to the discussion.
\llname{ll:variance_note}

%%%%%%%%%%%%%%
\reviewersection{2}


\begin{quote}
The authors have done a commendable job of responding to the reviewers' comments. The extended discussion and additional supplemental figures helps with the clarity of the manuscript. The results for the comparison with the discrete stepping-stone model were useful - though see below. The authors have also provided sufficient detail justifying their choice of model parameters.
\end{quote}

\begin{quote}
I don't think I have any critical remaining issues or suggestions, but I did have a couple of thoughts. The authors might just want to think about them.
\end{quote}

Thanks for the encouraging words and helpful points,
as well as for your mindfulness of the burden of additional analyses.

\begin{point}{}
I appreciate the appendix comparing summary statistics in the continuous and stepping-stone models, although I must confess I am now even more uncertain than I was before about which models are more appropriate. I get that the coalescent models make assumptions that are not directly interpretable in terms of observable parameters (although are ``interactions between geographically distant parts of the species range'' really biologically unrealistic?). This forward model is more interpretable, in that sense, but does it actually produce more realistic data? I guess that's a question for future work, but that question was one of the main things I came away from this paper with.
\end{point}

\reply{
This is a good point. We have added additional discussion noting that while our continuous model has some advantages over discrete models, the best model for any empirical system will inevitably depend on the life history of the organism in question. \llname{bestmodelcaveat}
}

\begin{point}{\revref}
 ``PCA is unable to capture''. Is this a sample size effect? i.e. the precision of the estimates of the PCs depends on sample size, so it's possible that you just need a larger sample to estimate the PCs when you have weak structure. Or is it something else?
\end{point}

\reply{
We have added the caveat ``given the sample sizes tested'' to the above statement. \revref
}

\begin{point}{\revref}
 Is Khera et al (2018) really the right reference for the idea of polygenic scores? They have been around for a lot longer than that.
\end{point}

\reply{Yes this was an oversight. We now cite Wray et al. 2007 for individual disease risk prediction from GWAS estimates and the 
first, to our knowledge, PGS application (International Schizophrenia Consortium, 2009).
}

\begin{point}{\revref}
 I don't know that the results of Berg and Sohail et al. are indicating a fundamental issue with PC correction, rather than a problem with meta-analysis.
\end{point}

\reply{
    We think \citet{Berg2018} shows that population structure leaves signals in GWAS that resemble those of polygenic selection, and given that this was most apparent in GIANT data employing PCA corrections, indicates a problem with the GWAS that make up the GIANT study at least as much as any issues with meta-analysis. 
}

\begin{point}{\llname{ldcaveats}}
 Is the result of Martin et al. really due to population stratification? There seem to be a number of factors affecting transferability of PRS (LD differences, effect size heterogeneity, differences in causal variants etc..), but I don't know that anyone thinks residual structure is the major one, particularly for eg. UKB summary stats.
\end{point}

\reply{This is a good point -- differences in allele frequency and LD structure certainly have significant effects on PRS transferability (at least when the GWAS is based on SNP assays rather than whole-genome sequencing). We think the population structure issue is important to point to here because it will not be solved by simply generating more detailed genomic data from existing samples, but have added a caveat noting these other important potential sources of error at the head of that sentence. \llname{ldcaveats}
}

\begin{point}{``population genetics flavor'' \revref}
I don't understand what this means?
\end{point}

\reply{
We meant to contrast this sort of ``single-species'' model,
common in population genetics,
to a more ``ecological'' model involving explicit modeling or resources and/or other species.
We've reworded this. \revref
}

\begin{point}{\revref}
``process-driven descriptions of ancestry and/or more generalized unsupervised methods''. Aren't these two things opposite? That's ok, but it sounds a bit like you have no idea what would help...
\end{point}

\reply{
They certainly entail research in somewhat different directions, but you are correct that we don't know which is the best way to go. 
}


\begin{point}{}
 Fix citation: Peter L. Ralph and Ashander ??? on pg. 5, line 216
\end{point}

\reply{
corrected - thanks. 
}

\begin{point}{}
'there' written twice on pg. 27, line 1033
\end{point}

\reply{
thanks for catching this, corrected. 
}




%%%%%%%%%%%%%%
\reviewersection{3}


\begin{quote}
I am fairly pleased overall with the revised version of this article. I commend the authors for the substantial amount of work conducted. This contribution now convincingly demonstrates how important it is to properly account for population structure when individuals are distributed along a spatial continuum.
\end{quote}

Thanks. We are also fairly pleased with it, overall.
\plr{this is probably too much =)}

\begin{point}{}
A couple of minor points about the forward model are still puzzling me though. First of all, in one dimension, the variance of $L(o)$, the random variable corresponding to the offspring's spatial position, given $l(p)$, the parent's position, is equal to $E((L(o)-l(p))^2):=\sigma^2$. Hence, $\sigma$ is here the square root of the expected squared Euclidean distance between parents and offspring. In general, assuming isotropic migrations, $\sigma$ is the square root of the expected squared Euclidean distance taken along any dimension. It seems to me that this definition is distinct from the mean parent-offspring distance, as stated by the authors.
\end{point}

\reply{
Good point here.
We did intend to define $\sigma$ as is usual in the literature,
so have corrected the wording
and now direct readers to \citep{Rousset1997} for discussion of how these parameters are defined in one- and two-dimensional habitats. \llname{axialdist}
}

\begin{point}{}
More importantly, I am still a bit confused about the reproduction scheme. My understanding is that, once mating takes place, the two parents involved (noted as P(1) and P(2)) each produce a Poisson number of offspring. The location of each offspring is determined by a Gaussian distribution of covariance matrix $\sigma^2 I$, centered on the location of the parent that ``produced' this offspring. Let $O(1,i)$ and $O(2,j)$ be the i-th and j-th offspring produced by $P(1)$ and $P(2)$ respectively. Now, if I am not mistaken, half of the genetic material of O(2,j) comes from P(1) (the other half coming from P(2)) while O(1,i) receives one chromosome from P(2) (and another one from its ``true' parent, i.e., P(1)). Hence, when it comes to the parent-offspring distance as measured from their genetic material, half of the chromosomes have a ``small' dispersal distance, equal to that of the parent-offspring distance, while the other half have a ``large' distance roughly equal to the parent-offspring distance plus the distance between mates.

It is important that the authors clarify this last point because most results presented in this study rely on Wright's neighborhood size, which derives from a model where dispersal of individuals and chromosomes follow well-defined spatial dynamics. A sensible comparison of neighborhood sizes requires that dispersal (along with effective population density) is quantified in a coherent way across different models.

Additionally, the neighborhood size as defined by the authors (see \llname{ll:nw_defn}) is a function of the census density rather than the effective density of individuals. I am therefore not fully convinced that the range of values for this parameter is comparable to that given in Table 1.
\end{point}

\reply{
    You are entirely correct here; apologies for the confusion.
    To clarify the definition of $\sigma$ and discuss these issues somewhat,
    we have added a new paragraph at \llname{ll:sigma_effective}.
    As we discuss there, we still calculate $N_W = 4 \sigma^2 \rho$,
    where $\sigma$ is the mean mother-child dispersal distance
    (not including the distance between mates)
    and $\rho$ is the census population density.
    This is not the quantity that is predicted
    to best describe geographic patterns of genetic variation,
    which would use \emph{effective} versions of both $\sigma$ and $\rho$.
    This does not bother us for a few reasons.
    First, we do not attempt to match statistics in our simulations
    to theoretical results (e.g., the Wright--Malec\'ot formula).
    Instead, we use $N_W$ as a proxy for ``strength of spatial structure'',
    and in these simulations, the ``effective'' $N_W$ certainly has a monotonic
    relationship with the $N_W$ we compute.
    The remaining question is whether, as the reviewer asks,
    the actual values of $N_W$ are comparable to those computed in the wild.
    This depends, of course, on how those values were estimated in wild populations.
    In our opinion, a straightforward estimate of $N_W$
    using observed dispersal distances and census density or reproductive adults
    (similar to how we compute $N_W$ for simulations)
    is likely more robust and reflective of modern demography
    than most methods to estimate ``effective'' vesions.
    
    Furthermore, our imprecisions roughly cancel out.
    As now noted in the text, an average of one and two dispersal distances
    is a better model for the mean parent-offspring distance;
    this would give us a value of $\sqrt{3/2} \sigma$,
    so by using only $\sigma$ we have lowered $N_W$ by a factor of about 2/3.
    However, effective population density is lower than census density
    by a factor of also around 2/3,
    which would cancel the first.
    So, we are fairly confident that our values of $N_W$ match the theoretical versions
    to within the precision we require (which is only order-of-magnitue, really).

These are an important point set of points on relating neighborhood sizes from different models which we have aimed to clarify. In our simulation the x axis displacement of a locus along any link in the pedigree is normally distributed with standard deviation $\sigma$ for the ``maternal'' material (i.e. deriving from the parent from which dispersal is calculated), but for the ``paternal'' parent it is the sum of two independent normal draws each with with standard deviation $\sigma$. Averaging over a random series of maternal and paternal transmissions (because our simulated genome is autosomal), variance in axial position across pedigree links is then $(3/2)\sigma^2$ and the expected displacement $\sqrt{3/2}\sigma$. So from the perspective of genes sampled at the present time our neighborhood size calculations are lower than those in models like those described in \cite{Rousset1997} by a factor of (3/2) because of differences in the breeding model. 

This inflation in neighborhood size is then partially offset by our use of census rather than ``effective'' population size when calculating population density, because $N_e$ of our spatial model estimated from $\pi$ in random-mating spatial models is roughly 2/3 the census size. We chose this method for density and sigma calculations because we sought to parameterize our model with values that could be observed in real populations. We also find the concept of ``effective population size'' in this context even more confusing than usual -- as we show, spatial processes in both evolution and sampling radically shift the distribution of coalescence times and the variance in offspring numbers in ways that mimic the effects of classic $N_{e}$ scaling in random-mating models, but are driven by fundamentally different processes.

Thus relative to models like \cite{Rousset1997} our definition of $\sigma$ is probably too low and $\rho$ probably too high such that the neighborhood sizes we show in plots are roughly consistent with previous theoretical work; while relative to neighborhood sizes calculated from mark-recapture studies measuring parent-offspring distances our figures should be consistent to the extent that the breeding and dispersal system matches our simulation. We have expanded the caveats at the head of the discussion paragraph interpreting empirical studies noting that definitions of $\rho$ in ecological vs genetic studies mean results should be interpreted with caution \llname{rhocaveat}; however, as we mainly interpret these figures in terms of order-of-magnitude differences (roughly less than 100, 100-1000, and over 1000) we think our results are general enough to be robust to these modeling differences while still giving a useful yardstick. Indeed for many of the empirical studies the range of NS values given is larger than the differences we expect from the parameter definitions described above. We also think that our study of lattice simulations in appendix 1 demonstrates nicely how qualitatively similar simulations can lead to quite different patterns of genetic diversity at a given neighborhood size as a result of differences in the underlying breeding model. 

\cjb{CJ: let's meet and talk again and talk about what edits to make here. I'm not sure what the best way to go is.}

}

\ak{ADK: I made some small edits but I'm pretty happy about how it reads}

\begin{point}{}
Furthermore, the neighborhood size estimates obtained for Bebicium vittatum derive from the analysis of a one-dimensional habitat. The product $\sigma^2 \rho$ is thus expressed in this particular case as a number of individuals *per unit of space* (in two dimensions, this product is expressed as a number of individuals). Hence, the unit of space matters a lot here, prohibiting the direct comparison with other figures in the same table or with the simulations conducted in this study.
\end{point}

\reply{
Good point, we have removed Bebicium vittatum from table 1.  
}


\begin{point}{}
Finally, Leblois, Estoup and Streiff (2006) Molecular Ecology conducted a large simulation studies with a focus very close to that of the present study. In particular, they investigated the impact of sampling on the inference of summary statistics commonly used in population genetics. It would probably be relevant to compare some of the results presented in Leblois et al. to that put forward here.
\end{point}

\reply{
Thank you for bringing this study to our attention. We have added it to the citations in the demographic inference section. 
}



%%%%%%%%%%%%%%
\reviewersection{4}


\begin{quote}
The authors have thoroughly addressed nearly all of my comments in their timely manuscript, and I am satisfied with their responses.
\end{quote}

Thank you very much!

\begin{point}{}
Regarding Table S1, I would still suggest adding a column about interpretation for those variables and evaluated with respect to the simulated results.
\end{point}

\reply{
Thank you for your comments. We chose to evaluate a few key statistics in detail in the main text and then provide an interpretation rooted in how the marginal genealogies of the tree sequence interact with space, and hope that this will assist readers when examining other statistics shown. 
}


